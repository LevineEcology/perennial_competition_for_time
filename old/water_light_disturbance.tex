% Created 2022-10-06 Thu 10:10
% Intended LaTeX compiler: pdflatex
\documentclass[11pt]{article}
\usepackage[utf8]{inputenc}
\usepackage[T1]{fontenc}
\usepackage{graphicx}
\usepackage{longtable}
\usepackage{wrapfig}
\usepackage{rotating}
\usepackage[normalem]{ulem}
\usepackage{amsmath}
\usepackage{amssymb}
\usepackage{capt-of}
\usepackage{hyperref}
\author{Jacob}
\date{\today}
\title{Competition for water and light in perennial plants with regular disturbance}
\hypersetup{
 pdfauthor={Jacob},
 pdftitle={Competition for water and light in perennial plants with regular disturbance},
 pdfkeywords={},
 pdfsubject={},
 pdfcreator={Emacs 27.2 (Org mode 9.6)}, 
 pdflang={English}}
\begin{document}

\maketitle
\tableofcontents



\section{A possibly contrived, but tractable case}
\label{sec:org10e6833}

\subsection{Setup}
\label{sec:orge2f2e97}

Consider four species each assigned one of two water (early vs. late) and one of two light (up vs. out) strategies. Species are labeled with two subscripts: the first describes its light strategy while the second describes its light strategy. Thus, for example, species [1,2] invests in height growth and drought tolerance. The two water strategies are defined by each species' critical soil water content, or \(W^*\), such that \(W_1^* > W_2^*\) while the light strategies are defined by the allometric parameter \(\phi\) from \cite{dettoMaintenanceHighDiversity2021a}, such that the leaf area growth of a species with light strategy 1 is given by \(\frac{1}{a + b \phi_1} G^L\) and \(\phi_1 > \phi_2\).

In the model, stand-replacing disturbance occurs at a fixed interval, \(t_0\), and soaking rains occur at a fixed interval \(\tau_0\). These rains return the soil to field capacity, \(W_0\). Throughout, the variable \(t\) refers to time since disturbance, while the variable \(\tau\) refers to time since the last soaking rainstorm. Though plants' growth rates vary at the \(\tau\) scale due to periodic water limitation when \(W < W_i^*\), lets assume for the sake of mathematical tractability that a plant grows at its average growth rate for the duration of each inter-rain interval. Thus, a plant's average growth rate, \(G\), is a function of the time it spends under water limitation, \(\tau_0 - \tau_i\), where \(\tau_i\) is the (inter-rain) time at which species with water strategy \(i\) become water limited (i.e. when \(W(\tau) = W_i^*\)).

Lets start with the case where all four possible combinations of the two light and water strategies are represented in a community of four species (i.e. the four species are given the subscripts [1,1], [1,2], [2,1], and [2,2]). Next assume that the productivity of the system is high, and thus the canopy closes quick enough relative to the disturbance interval that we can approximate the dynamics by assuming the canopy closes immediately after disturbance. Thus, the species which specialize in out growth (have smaller \(\phi\)) begin falling into the understory immediately after disturbance. Because we have assumed instantaneous canopy closure, the total amount of canopy per unit area is equal to 1 for the duration of the inter-disturbance interval.

\subsection{Within-disturbance-period dynamics: shutoff time, growth rate, and leaf area expressions}
\label{sec:orgd3893c2}

If we assume that transpiration is proportional to leaf area by the constant \(E\), then the time until the species with the early water strategy cease growth is constant and given by:

\begin{equation}
    \tau_1 = \left[\frac{W_0 - W_1^*}{E (L+1)}\right]^{\frac{1}{L+1}}
\end{equation}

Thus, because canopy area remains at 1 until the next disturbance, species with an early water strategy grow at a rate of \(G(\tau_1)\) for the duration of their lifespan. The relationship between growth rate and \(\tau\) is given by:

\begin{equation}
    G_j(\tau) = \frac{1}{\tau_0}\left[ (g_{1,j} + g_{2,j}) \tau - g_{2,j} \tau_0 \right]
\end{equation}

and the leaf area, \(\Lambda\), of species with water strategy 1 and light strategy \(i\) at time \(t\) is given by

\begin{equation}
    \Lambda_{i,1}(t) = N_{i,1} k_i G_j(\tau_1)^L t^L
\end{equation}

in the absence of light competition, and where \(k_i = \left(\frac{1}{a + b \phi_i}\right)^L\). If species compete for light, as we assume they do, then the two species with light strategy 2 begin folding into the understory after the canopy first closes (in our case at \(t=0\)). This means that the total leaf area of species with light strategy 2 (the shorter species) is given by:

\begin{equation}
    \sum_{j=1}^2 \Lambda_{2,j}(t) = 1 - \sum_{j=1}^2 \Lambda_{1,j}(t)
\end{equation}

for all \(t > 0\). Now, note that the total leaf area of the species with water strategy 2 (late species) is equal to

\begin{equation}
    \sum_{i=1}^2 \Lambda_{i,2}(t) = \mathcal{L}_1(t) \sum_{j=1}^2 \Lambda_{1,j}(t) + \mathcal{L}_2(t) (1 - \sum_{j=1}^2 \Lambda_{1,j}(t))
\end{equation}

\begin{equation*}
     = (\mathcal{L}_1(t) - \mathcal{L}_2(t))\sum_{j=1}^2 \Lambda_{1,j}(t) + \mathcal{L}_2(t)
\end{equation*}

where,

\begin{equation}
    \mathcal{L}_i(t) = \frac{\Lambda_{i,2}(t)}{\Lambda_{i,2}(t) + \Lambda_{i,1}(t)}
\end{equation}

If \(\tau_2\) varies in time, then so does \(\mathcal{L}_i\), and the problem is not analytically tractable, as \(\Lambda_{i,2}\) is nonlinearly dependent on all past values of \(\tau_2\). While numerical solutions can still be found relatively inexpensively, a numerical problem is far less satisfying than an analytical case.

Thankfully, an analytical solution can be found for the special case where \(\mathcal{L}_i\) is fixed within each inter-disturbance period. \(\mathcal{L}_i\) is fixed when \(\mathcal{L}_1 = \mathcal{L}_2\). We now derive the conditions under which this equality holds:

Since already know that \(G_1(\tau_1)\) is fixed, we only need to determine the fate of \(G_2(\tau_2(t))\). To do so we begin with the first period, in which the initial values of \(\mathcal{L}_1\) and \(\mathcal{L}_2\), \(\mathcal{L}_1(0)\) and \(\mathcal{L}_2(0)\), are determined entirely by the initial sizes of the individuals and the relative population densities within each light strategy. We calculate \(\tau_2\) in the first inter-rain interval, \(\tau_2(0)\) by relating equation 5 to the water available between the first and second water strategy shutdown points.

\begin{equation}
   W^*_1 - W^*_2 = E \int_{\tau_1}^{\tau_2(0)} (\mathcal{L}_1(0) - \mathcal{L}_2(0))\sum_{j=1}^2 \Lambda_{1,j}(0) + \mathcal{L}_2(0) d\tau
\end{equation}

For the math to work out, we next need to assume that \(\mathcal{L}_1(0) = \mathcal{L}_2(0) = \mathcal{L}(0)\) during the inital timestep (note that this only needs to be assumed for the initial timestep of the first inter-disturbance interval). If all individuals of all species begin with the same initial size, \(\Lambda_0\) this is true if the ratios of population densities between species of differing water strategy are the same across light strategies. When \(\mathcal{L}_1(0) = \mathcal{L}_2(0) = \mathcal{L}(0)\), equation 7 becomes:

\begin{equation*}
     W^*_1 - W^*_2 = E \int_{\tau_1}^{\tau_2(0)} \mathcal{L}(0) d\tau
\end{equation*}

And after taking the integral and solving for \(\tau_2(0)\) we find:

\begin{equation*}
    \tau_2(0) = \tau_1 + \frac{W_1^* - W_2^*}{E \mathcal{L}(0)}
\end{equation*}

making the leaf area of species [i,2] at the end of the first interrain interval:

\begin{equation*}
    \Lambda_{i,2}(\tau_0) = N_{i,2} k_i G_2(\tau_2(0))^L \tau_0^L
\end{equation*}

Therefore,

\begin{equation*}
    \mathcal{L}_i(\tau_0) = \frac{N_{i,2} k_i G_2(\tau_2(0))^L \tau_0^L}{N_{i,2} k_i G_2(\tau_2(0))^L \tau_0^L + N_{i,1} k_i G_1(\tau_1)^L \tau_0^L}
\end{equation*}

which simplifies to

\begin{equation*}
     \mathcal{L}_i(\tau_0)= \frac{G_2(\tau_2(0))^L}{G_2(\tau_2(0))^L + X_i G_1(\tau_1)^L}
\end{equation*}

where \(X_i = \frac{N_{i,1}}{N_{i,2}}\). Thus, \(\mathcal{L}_1(\tau_0) = \mathcal{L}_2(\tau_0)\) when \(X_1 = X_2\). Remember that \(X_1 = X_2\) is the requirement for \(\mathcal{L}_1(0) = \mathcal{L}_2(0)\), which we have already assumed. In the next inter-rain interval, \(\tau_2(\tau_0)\) is given by:

\begin{equation*}
    \tau_2(\tau_0) = \tau_1 + \frac{W_1^* - W_2^*}{E \mathcal{L}(\tau_0)}
\end{equation*}

and thus:

\begin{equation*}
    \mathcal{L}_i(2\tau_0) = \frac{\left[G_2(\tau_2(0)) + G_2(\tau_2(\tau_0)) \right]^L }{\left[G_2(\tau_2(0)) + G_2(\tau_2(\tau_0)) \right]^L +X_i G_1(\tau_1)^L }
\end{equation*}

In general:

\begin{equation}
    \mathcal{L}_i(q\tau_0) = \frac{\left[ \sum_{j=0}^q G_2(\tau_2(j \tau_0))\right]^L }{\left[ \sum_{j=0}^q G_2(\tau_2(j \tau_0))\right]^L + X_i G_1(\tau_1)^L q^L}
\end{equation}

where \(q\) is the number of inter-rain intervals which have occurred since the last disturbance.

\begin{equation}
    \tau_2(q\tau_0) = \tau_1 + \frac{W_1^* - W_2^*}{E \mathcal{L}(q\tau_0)}
\end{equation}

From equations 8 and 9 we now see that

\begin{equation*}
    \mathcal{L}_1(t) = \mathcal{L}_2(t)
\end{equation*}

for all \(0 < t < t_0\).

\subsection{Between disturbance dynamics: population growth rate}
\label{sec:org942ab33}

For now lets supposed the plants in this model are perennial shrubs which produce seeds proportional to their leaf area by a constant \(f\). Assume these seeds remain dormant in the soil until the next disturbance at which point they germinate with probability \(g\). For now lets also assume that seed survival in the seedbank is 1, however this assumption can be relaxed without much added complexity. In such a system, the total reproduction of short species during an inter-disturbance interval is given by:

\begin{equation}
    N_{2,1} = F \int_{0}^{t_1} (1-\mathcal{L}_2(t)) (1 - \Lambda_{1}(t))dt
\end{equation}

\begin{equation}
    N_{2,2} = F \int_{0}^{t_1} \mathcal{L}_2(t) (1 - \Lambda_{1}(t))dt
\end{equation}

where \(F = fg\) and \(t_1\) is the time at which the taller species close the canopy above the shorter species. Because we assume an absolute height hierarchy, the species with light strategy 2 (shorter species) will be completely in the understory after the species with light strategy 1 close the canopy above. \(t_1\) is calculated from the following expression:

\begin{equation*}
    1 = k_1 N_{1,2} \left[X_1 G_1(\tau_1)^L t_1^L + \left( \sum_{j=0}^{\left\lfloor \frac{t_1}{\tau_0}\right\rfloor} G_2(\tau_2(j \tau_0)) + G_2\left(\tau_2\left(\left\lceil \frac{t_1}{\tau_0}\right\rceil\right)\right) \left(t_1 \pmod{\tau_0} \right) \right)^L \right]
\end{equation*}

Because of the nonlinear dependence of \(G_2\) on \(\tau_2(t)\) and \(\mathcal{L}(t)\), solving for \(t_1\) remains numerical problem, and thus so does the population dynamic equations, unless \(\mathcal{L}(0) = \frac{G_2(\tau_2(\tau_0))^L}{G_2(\tau_2(\tau_0))^L + X_i G_1(\tau_1)^L}\), where \(\tau_2'\) is an unknown, fixed quantity. If this is true, then \(\mathcal{L}(t)\) remains constant for the entire inter-disturbance interval, and thus so does \(\tau_2\), making the system tractable. Under what conditions
\end{document}
